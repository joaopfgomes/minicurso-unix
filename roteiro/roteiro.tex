%!TEX encoding = UTF-8
\documentclass[a4paper,10pt,titlepage,twosided]{book}
\usepackage[utf8]{inputenc}
\usepackage{amssymb,amsmath}
\usepackage{array}
\usepackage{algorithm}
\usepackage{algorithmic}
\usepackage{graphicx}
%\usepackage{subfig}
\usepackage{lscape}
\usepackage{xspace}
\usepackage{setspace}
\usepackage{scale}
\usepackage{multirow}
\usepackage{booktabs}
\usepackage{pdfpages}
\usepackage{pdflscape}
\usepackage{subfig}
\usepackage{array}
\usepackage[sort&compress, numbers]{natbib}
\usepackage{changepage}
\usepackage{url}
\usepackage{listings}
\usepackage{soul}
\usepackage{verbatimbox}
\usepackage{chngcntr}
\counterwithout{section}{chapter}

\usepackage[portuguese, english]{babel}
%\usepackage{anysize}
%\marginsize{0cm}{0cm}{0cm}{0cm}
\usepackage[tmargin=2cm,rmargin=2.8cm,lmargin=1cm, bmargin=3cm]{geometry}
\usepackage[Sonny]{fncychap}
\usepackage{drop}

 \ChTitleVar{\LARGE\sffamily\bfseries}
%    \ChRuleWidth{0.5pt}
    
\let\origdoublepage\cleardoublepage
\newcommand{\clearemptydoublepage}{%
  \clearpage
  {\pagestyle{empty}\origdoublepage}%
}

\newtheorem{definition}{Definition}
\renewcommand{\algorithmicrequire}{\textbf{Input:}}
\renewcommand{\algorithmicensure}{\textbf{Output:}}

%\newcommand{\drop}[1]{{\fontsize{5cm}{1em}\selectfont\textsf #1}}

\pagestyle{myheadings}
\usepackage{geometry}
%\geometry{bindingoffset=1cm}
\setlength{\evensidemargin}{-2cm}
\setlength{\oddsidemargin}{-1cm}
%\let\tmp\oddsidemargin
%\let\oddsidemargin\evensidemargin
%\let\evensidemargin\tmp
%\reversemarginpar

\title{Minicurso de UNIX}
\author{Leonardo Bezerra}
\begin{document}

\newcommand{\rights}{\footnote{Autor: Leonardo Bezerra.}}

\definecolor{light-green}{RGB}{234, 247, 178}
\sethlcolor{light-green}
\newcommand{\gmm}{\texttt{g++}\xspace}
\newcommand{\git}{\texttt{git}\xspace}

\newenvironment{code}
{\verbbox}
{\endverbbox\par\bigskip\colorbox{light-green}{\parbox{\textwidth}{\theverbbox}}\par}

\newenvironment{incode}
{\verbbox}
{\endverbbox\par\colorbox{light-green}{\parbox{\textwidth}{\theverbbox}}\par}

\setcounter{page}{1}
\setcounter{chapter}{1}
\chapter*{Minicurso de UNIX\rights}

\section{Roteiro básico}

Abaixo você encontrará a descrição de uma série de tarefas triviais presentes no cotidiano de quem trabalha com sistemas UNIX. Caso tenha dúvida na utilização de algum comando, consulte seu manual ou a tabela de comandos úteis encontrada neste link: \url{http://cheatsheetworld.com/programming/unix-linux-cheat-sheet/}.

\subsubsection{Diretórios}

\begin{enumerate}
\medskip
\item Abra o terminal e verifique em qual pasta ele é inicializado. 
\item Crie um diretório \textsf{minicursos} e um subdiretório \textsf{unix}.
\item Use o comando \texttt{ls} para verificar que o diretório e o subdiretório foram criados.
\item[] \textbf{Extra -- é possível criar um diretório e seu subdiretório com apenas uma chamada ao comando \texttt{mkdir}. Também é possível listar um diretório recursivamente utilizando o comando \texttt{ls}.}
\end{enumerate}

\subsubsection{Arquivos}

\begin{enumerate}
\medskip
\item Com o auxílio de um editor de texto, crie dois arquivos de nome \textsf{arquivo1.txt} e \textsf{arquivo2}, contendo, respectivamente, as frases ``\emph{Este é o arquivo 1.}" e ``\emph{Este é o arquivo 2.}".
\item Mova ambos os arquivos para o diretório \textsf{minicursos}.
\item Crie um arquivo vazio \textsf{vazio.txt} no subdiretório \textsf{unix}.
\item Copie o arquivo \textsf{arquivo2} para o subdiretório \textsf{unix}.
\item Renomeie o arquivo \textsf{arquivo2} contido no subdiretório \textsf{unix} para \textsf{nao-vazio}.
\end{enumerate}

\subsubsection{Arquivos e fluxos}

\begin{enumerate}
\medskip
\item Baixe o arquivo \textsf{exemplo.in} disponibilizado no SIGAA.
\item Crie um link simbólico chamado \textsf{exemplo} na pasta minicursos para o arquivo baixado no item anterior.
\item Visualize o conteúdo de \textsf{exemplo} sem usar um editor de texto.
\item Sem usar um editor de texto, copie as 5 primeiras linhas do arquivo apontado por \textsf{exemplo} para um novo arquivo \textsf{mini-exemplo.txt}, que deve ser criado dentro do subdiretório \textsf{unix}.
\item Sem usar um editor de texto, adicione as 5 últimas linhas do arquivo apontado por \textsf{exemplo} ao arquivo \textsf{mini-exemplo.txt}.
\item Gere um arquivo \textsf{mini-exemplo2.txt} idêntico ao arquivo \textsf{mini-exemplo.txt} sem usar os comandos \texttt{cp}, \texttt{mv}, \texttt{head}, \texttt{tail} ou um editor de texto. Assegure que os arquivos sejam idênticos sem usar um editor de texto.
\end{enumerate}

\subsubsection{Compressão de arquivos}

\begin{enumerate}
\medskip
\item Comprima o arquivo \textsf{mini-exemplo.txt} usando compressão Gzip.
\item Compare os tamanhos dos arquivos \textsf{mini-exemplo.txt.gz} e \textsf{mini-exemplo2.txt}.
\item Acesse o diretório pai do diretório \textsf{minicursos}.
\item Crie um pacote \textsf{tar} sem compressão chamado \textsf{minicursos.tar}, contendo o diretório \textsf{minicursos}.
\item Crie um pacote \textsf{tar} com compressão de arquivos Gzip chamado \textsf{minicursos.tar.gz}, contendo o diretório \textsf{minicursos}.
\item Crie um pacote \textsf{tar} com compressão de arquivos XZ chamado \textsf{minicursos.tar.xz}, contendo o diretório \textsf{minicursos}.
\item Crie um pacote comprimido Zip chamado \textsf{minicursos.zip}, contendo o diretório \textsf{minicursos}.
\item Compare o tamanho dos diferentes arquivos de nome-base \textsf{minicursos}.
\item[] \textbf{Extra -- para ver o tamanho de um arquivo, use uma das opções do comando \texttt{ls}. Para ver o tamanho de um diretório, consulte o manual do comando \texttt{du}.}
\end{enumerate}

\subsubsection{Busca}

\begin{enumerate}
\medskip
\item Busque no arquivo exemplo as palavras ``\emph{teste}" e ``\emph{testa}", sem usar um editor de texto.
\item Acesse sua pasta home (não confunda com o diretório \texttt{/home}).
\item Localize o arquivo cujo nome contenha a palavra \textsf{exemplo}.
\item Filtre a lista de ocorrências encontradas no item acima para mostrar apenas arquivos cuja extensão seja \texttt{txt}.
\item Busque, em todas as ocorrências encontradas no item acima, a palavra ``\emph{Este}".
\item[] \textbf{Extra -- para executar o item 5, utilize o comando \texttt{xargs}.}
\end{enumerate}

\subsubsection{Manipulação de arquivos}

\begin{enumerate}
\medskip
\item Usando o arquivo \textsf{arquivo2} contido no diretório \textsf{minicursos} como base, gere um arquivo de nome \textsf{recortado.txt} no subdiretório \textsf{unix} contendo a frase ``\emph{Este é o 2.}".
\item Usando o arquivo \textsf{arquivo2} contido no diretório \textsf{minicursos} como base, gere um arquivo de nome \textsf{esfacelado.txt} no subdiretório \textsf{unix} contendo a frase ``\emph{Este arquivo}".
\item Usando o arquivo \textsf{arquivo2} contido no diretório \textsf{minicursos} como base, gere um arquivo de nome \textsf{tabulado.txt} no subdiretório \textsf{unix}, contendo o mesmo conteúdo de \textsf{arquivo2}, porém com separação de palavras por tabulação em vez de espaço.
\item Considerando o arquivo apontado por \textsf{exemplo}, liste em ordem alfabética os comandos aprendidos no minicurso.
\item Conte a quantidade de comandos identificados no item anterior.
\item[] \textbf{Extra -- para executar o item 3, utilize o comando \texttt{tr}.}
\end{enumerate}

\subsubsection{Processos}

\begin{enumerate}
\medskip
\item Abra um editor de texto de sua preferência.
\item Identifique o código de processo da instância aberta do editor de texto.
\item Mate a instância aberta do editor de texto.
\item Liste, em ordem alfabética e sem repetições, os usuários que têm processos ativos no sistema.
\item Conte a quantidade de processos ativos iniciados pelo usuário \textsf{root}.
\end{enumerate}

\subsubsection{Permissões}

\begin{enumerate}
\medskip
\item Crie um arquivo de nome \textsf{arquivo\_restrito} e configure suas permissões para que ninguém possa utilizá-lo para leitura, escrita ou execução. Verifique se você consegue realizar alguma destas operações.
\item Crie um diretório de nome \textsf{dir\_restrito} e configure suas permissões para que ninguém possa visualizar seu conteúdo. Verifique se você consegue visualizar seu conteúdo após esta operação.
\item Altere as permissões do diretório \textsf{dir\_restrito} para que seu proprietário (você) possa navegá-lo, mas não possa criar arquivos nele.
\end{enumerate}

\subsubsection{Executáveis}

\begin{enumerate}
\item Baixe o arquivo \textsf{exemplo.sh}, disponível no SIGAA. Configure suas permissões para que você possa executá-lo. Teste sua execução.
\item Configure a variável \texttt{PATH} para poder executar o script acima sem precisar digitar seu caminho. Faça esta configuração de forma que apenas a sessão em uso do terminal tenha sua configuração alterada.
\item Inicie uma nova sessão do terminal (nova aba ou janela) e configure a variável \texttt{PATH} para poder executar o script acima sem precisar digitar seu caminho, mas de forma que qualquer nova sessão do terminal seja afetada. Teste esta configuração na tela já em uso do terminal e também em uma nova janela.
\end{enumerate}

\end{document}


